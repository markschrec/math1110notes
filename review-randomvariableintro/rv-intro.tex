\documentclass[14,fleqn]{article}
\usepackage{amsmath}
\usepackage{amssymb}
\usepackage[top=.5 in,left=.5 in,right=.5 in,bottom=.5 in]{geometry}
\usepackage{enumerate}
\usepackage{ mathrsfs }
\usepackage{graphicx}
\usepackage{pgf,tikz}
\usepackage{mathrsfs}
\usepackage{gensymb}
\usepackage{venndiagram}
\usepackage{enumitem}
\usetikzlibrary{arrows}

\pagenumbering{gobble}

\setlength{\parindent}{0 pt}
\setlength{\parskip}{1 ex}

\newcommand{\lcm}{\textnormal{lcm}}
\newcommand{\norm}{\triangleabove right}
\newcommand{\bfm}[1]{$\boldsymbol{#1}$}
\newcommand{\Z}{\ensuremath{\mathbb{Z}}}
\newcommand{\R}{\ensuremath{\mathbb{R}}}
\newcommand{\C}{\ensuremath{\mathbb{C}}}
\renewcommand{\wedge}[1]{\ensuremath{\langle #1 \rangle}}
\newcommand{\infsum}[1]{\ensuremath{\sum_{n=#1}^\infty}}
\newcommand{\defn}[1]{\textbf{\underline{#1}}}

%\begin{venndiagram3sets}[labelA=$S$,labelB=$T$,labelC=$U$]
%	\fillA
%	\fillOnlyC
%\end{venndiagram3sets}\\

%\begin{venndiagram2sets}[labelA=$S$,labelB=$T$]
%	\fillNotA
%	\fillNotB
%	\setpostvennhook{
%		\draw[] (labelAB) ++(0,-2.1) node {\raisebox{0pt}[0pt][0pt]{$(S\cap T)'$}};
%	}
%\end{venndiagram2sets}\\

\begin{document}
\section{More Review/Random Variable Intro}
We are going to spend the day doing more probability questions (and so more counting questions) but with some added notation we will use later.

Example: Suppose we flip a coin 6 times. Let $X$ be the number of heads that occurs. Find the probability we get exactly 2 heads. We can use the notation $P(X=2).$ Similarly, find $P(X=3).$ Now make a table showing all the possible values of $X$ and the corresponding probabilities.

This type of notation is what we will call a Random variable. It is a convinence that we will use for experiments with some sort of numerical outcome. On of the best uses is that we can treat $X$ algebraically. So we will ask questions like $P(X\ge 1)$ or $P((X-3)=0)$ both of which we can answer.\\

Example: A basketball player shoots freethrows with a percentage of 80\%. They shoot until they miss or make 5. Let $X$ be the number of shots they made. Find $P(X=2).$ Find $P(X=5).$ Make a table showing all possible values of $X$ and $P(X=k)$ for each one of them.\\

The data in this table is called a probability distribution. It encapsulates all outcomes of an experiment and their probabilities. Then finding the probability of more complicated outcomes is just adding table values. For example $P(X\ge 3)=P(X=3)+P(X=4)+P(X=5).$\\

Here is another situation where this idea can be very convinent to use.\\

Example: Suppose we roll 2 fair 6 sided dice. Let $X$ be the outcome of the first and $Y$ be the outcome of the second. Find
\begin{itemize}
	\item $P(X=4)$
	\item $P(X>Y)$
	\item $P(X+Y=5)$
	\item $P(X=Y)$
\end{itemize}
And describe each of these in words.\\


Answer more review questions if we have time.

\end{document}
