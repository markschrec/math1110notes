\documentclass[14,fleqn]{article}
\usepackage{amsmath}
\usepackage{amssymb}
\usepackage[top=.5 in,left=.5 in,right=.5 in,bottom=.5 in]{geometry}
\usepackage{enumerate}
\usepackage{ mathrsfs }
\usepackage{graphicx}
\usepackage{pgf,tikz}
\usepackage{mathrsfs}
\usepackage{gensymb}
\usepackage{venndiagram}
\usepackage{enumitem}
\usetikzlibrary{arrows}

\pagenumbering{gobble}

\setlength{\parindent}{0 pt}
\setlength{\parskip}{1 ex}

\newcommand{\lcm}{\textnormal{lcm}}
\newcommand{\norm}{\triangleabove right}
\newcommand{\bfm}[1]{$\boldsymbol{#1}$}
\newcommand{\Z}{\ensuremath{\mathbb{Z}}}
\newcommand{\R}{\ensuremath{\mathbb{R}}}
\newcommand{\C}{\ensuremath{\mathbb{C}}}
\renewcommand{\wedge}[1]{\ensuremath{\langle #1 \rangle}}
\newcommand{\infsum}[1]{\ensuremath{\sum_{n=#1}^\infty}}
\newcommand{\defn}[1]{\textbf{\underline{#1}}}

%\begin{venndiagram3sets}[labelA=$S$,labelB=$T$,labelC=$U$]
%	\fillA
%	\fillOnlyC
%\end{venndiagram3sets}\\

%\begin{venndiagram2sets}[labelA=$S$,labelB=$T$]
%	\fillNotA
%	\fillNotB
%	\setpostvennhook{
%		\draw[] (labelAB) ++(0,-2.1) node {\raisebox{0pt}[0pt][0pt]{$(S\cap T)'$}};
%	}
%\end{venndiagram2sets}\\

\begin{document}
\section{Section 6.4: Conditional probability and independence}
So far we have studied probability in its simplest form, where all events are unrelated and there are no conditions to consider. But this is often not the case and it can be more helpful to consider events that are related. 

Example: Suppose that a survey of 100 UVA students finds that 30 are first years, 35 are taking a math class, and 55 are not first years or taking a math class.
\begin{enumerate}[topsep=0pt]
	\item What is the probability a random student is a first year?
	\item What is the probability a random student is taking a math class?
	\item If a randomly selected student is a first year, what is the probability they are taking a math class?
\end{enumerate}

Let's look at the Venn diagram.
\begin{center}
	\begin{venndiagram2sets}[labelA=First Year,labelB=Math,labelOnlyA=10,labelOnlyB=15,labelAB=20,labelNotAB=55]
\end{venndiagram2sets}\\
\end{center}

So the probability a random student is a first year is $30/100$ while the probability a random student is taking a math class is $35/100.$ However, what happens when we know the student is a first year? Then we should restrict our attention to only the first years. How many are there? How many are taking a math class? Thus if the student is a first year then they probability they are taking a math class is $20/30.$\\

The previous is an example of \defn{conditional probability}. We say ``The probability of $E$ given $F$'' and write $P(E|F)$ when we want the probability of $E,$ when know $F$ has already occured. This essentially restricts our sample space to only $F.$ For this reason we get the following formula
\[
	P(E|F)=\frac{P(E\cap F)}{P(F)} \mathrm{ provided }P(F)\neq 0
\]
Note that if $P(F)=0$ then the conditional probability doen't make sense anyway since $F$ can't occur. 

Basic examples:\\
\begin{enumerate}[topsep=0pt]
	\item If a woman has 2 children, what is the probability she has 2 boys, given the first child is a boy?
	\item Flip a coin 3 times. What is the probability you get exactly 1 head given that the first flip is a head?
	\item If you roll 2d6, what is the probability you get a 10 given that the first roll is a 2?
\end{enumerate}

Notice that when we have equally probable outcomes, the term $n(S)$ always cancels. So if every outcome has equal probability then we get $P(E|F)=n(E\cap F)/n(F)$ which also enforces the idea that we are restricting our sample space to just $F.$

We can also rearange the first formula to get the following product rule. If $P(F)\neq 0$ and $P(E)\neq 0$ then
\[
	P(E\cap F)=P(F)P(E|F)=P(E)P(F|E)
\]
Makes sense as well because if we want $E$ and $F$ to happen then we need $E$ to happen, then have $F$ happen given that $E$ has already happened.

Example:
Suppose we have a deck of cards and are dealt 2 cards in order. Find the probability of the following events:
\begin{enumerate}[topsep=0 pt]
	\item The first card is red and the second card is black
	\item Both cards are aces
	\item The first card is a face card (J,Q,K) and the second card is not.
\end{enumerate}

To compute the probabilities we will use the product rule.
\begin{enumerate}
	\item Let $E=$first card is red and $F=$second card is black. Then $P(E\cap F)=P(E)P(F|E).$ The probability that the first card is red is 1/2. Now we consider the probability $P(F|E).$ If the event $E$ has already happend then there is one less red card in the deck. Thus there are 26 black cards and 51 cards total so $P(F|E)=26/51.$ Thus $P(E\cap F)=(1/2)(26/51).$
	\item The probability the first card is an ace is $1/13.$ The probability the second card is an ace given that the first is an ace is $3/51$ thus the total probability both are aces is $(1/13)(3/51).$ (Compare with the old way of computing probability)
\item If the first card is a face then that has probability $(12/52).$ Then after picking that card the probability of not a face is $(40/51)$ so we multiply to get the probability of both.
\end{enumerate}

This rule generalizes to more events
\[
	P(E_1\cap E_2\cap E_3)=P(E_1)P(E_2|E_1)P(E_3|E_2\cap E_1)
\]

What if events don't have any relation to each other. For example, what is the probability that it will rain tomorrow given that I played squash yesterday. These two events have no relation so we want $P(F|E)=P(F).$ This gives the following definition:\\
Two events, $E$ and $F$ are called \defn{independent} if $P(F|E)=P(F)$ and $P(E|F)=P(E).$ Equivalently, we say they are equivalent if $P(E\cap F)=P(E)P(F).$

Examples: Suppose we flip a coin 4 times in a row. Determine if the events $E$ and $F$ are independent in each case.
\begin{enumerate}
	\item $E=$first flip is a head, $F=$ second flip is a tail.
	\item $E=$first flip is a head, $F=$ get at least 1 tail.
	\item $E=$ first flip is a tail, $F=$ get at least 1 head and tail
\end{enumerate}

Solution:\\
\begin{enumerate}
	\item Clearly $P(E)=1/2.$ Now let's compute $P(F|E).$ If the first flip is a head then there are 8 outcomes. 4 of them have the second flip as a tail, so $P(F|E)=1/2=P(F).$ Thus $E$ and $F$ are independent.
	\item Again we get $P(E)=1/2$ and $P(F)=15/16.$ Now we see $P(E\cap F)$ is the probability the first flip is a head and there is at least 1 tail. There are 7 outcomes which satisfy this out of the possible 16 so $P(E\cap F)=7/16.$ Thus these events are not independent.
	\item $P(E)=1/2$ and $P(F)=14/16.$ There are 7 outcomes which satisfy $E$ and $F$ so $P(E\cap F)=7/16.$ Thus the events are independent.
\end{enumerate}

One last thing to not. A collection of events is called independent if \underline{any} subset $E_1,\dots,E_k$ satisfies
\[
	P(E_1\cap \cdots \cap E_k)=P(E_1)\cdots P(E_k)
\]
This is also refered to as pairwise independent.

Example: Consider rolling a fair d6. And consider the following events. $A=\{1,2,3\},B=\{2,3,4\},C=\{3,4,5,6\}.$ Show that $P(A\cap B\cap C)=P(A)P(B)P(C)$ but that the events are not independent.\\

Check $P(A\cap B)=1/3\neq P(A)P(B).$


\end{document}
