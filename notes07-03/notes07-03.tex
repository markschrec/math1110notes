\documentclass[14,fleqn]{article}
\usepackage{amsmath}
\usepackage{amssymb}
\usepackage[top=.5 in,left=.5 in,right=.5 in,bottom=.5 in]{geometry}
\usepackage{enumerate}
\usepackage{ mathrsfs }
\usepackage{graphicx}
\usepackage{pgf,tikz}
\usepackage{mathrsfs}
\usepackage{gensymb}
\usepackage{venndiagram}
\usepackage{enumitem}
\usetikzlibrary{arrows}

\pagenumbering{gobble}

\setlength{\parindent}{0 pt}
\setlength{\parskip}{1 ex}

\newcommand{\lcm}{\textnormal{lcm}}
\newcommand{\norm}{\triangleabove right}
\newcommand{\bfm}[1]{$\boldsymbol{#1}$}
\newcommand{\Z}{\ensuremath{\mathbb{Z}}}
\newcommand{\R}{\ensuremath{\mathbb{R}}}
\newcommand{\C}{\ensuremath{\mathbb{C}}}
\renewcommand{\wedge}[1]{\ensuremath{\langle #1 \rangle}}
\newcommand{\infsum}[1]{\ensuremath{\sum_{n=#1}^\infty}}
\newcommand{\defn}[1]{\textbf{\underline{#1}}}

%\begin{venndiagram3sets}[labelA=$S$,labelB=$T$,labelC=$U$]
%	\fillA
%	\fillOnlyC
%\end{venndiagram3sets}\\

%\begin{venndiagram2sets}[labelA=$S$,labelB=$T$]
%	\fillNotA
%	\fillNotB
%	\setpostvennhook{
%		\draw[] (labelAB) ++(0,-2.1) node {\raisebox{0pt}[0pt][0pt]{$(S\cap T)'$}};
%	}
%\end{venndiagram2sets}\\

\begin{document}
\section{Section 7.3: Binomial Distribution}

A binomial trial is an experiment with 2 possible outcomes. This is the simplest (non-trivial) experiment. We will typically denote the two outcomes as ``success'' and ``failure''. Some examples include whether a treatment is effective for a condition, testing a product as defective or not, or winning a game.

When considering binomial trials we will denote the probability of succes as $p$ and the probability of failure as $q.$ Since this is a probability experiment we will have $0\le p\le 1$ and $p+q=1.$ In most practical examples $p$ will not be 0 or 1 because these are uninteresting experiments.

Now consider the following experiment. We take some binomial trial and repeat it $n$ times. We assume each binomial trial is independent so the probabilities do not affect on another. The main problem we have to answer is what is the probability of a certain number of successes.\\

Now let's be a little more formal. Suppose we have $n$ independent binomial trials, each of which has a probability of success of $p.$ Let $X$ denote the number of sucesses in the $n$ trials. What is the probability distribution of $X?$

How do we find $P(X=k)?$ What does a typical sequences of successes and failures look like. What is the probability of a sequence with exactly $k$ successes? How many sequences have $k$ successes. This gives us the following result
\[
	P(X=k)=\binom{n}{k}p^kq^{n-k}
\]

Example: Suppose we roll a fair standard die 5 times and we want to count the number of 6's we get. What is the probability we get 3 6's. 

Before we do more examples, lets verify this is a probability assignment.
What is the sum $P(X=0)+\cdots P(X=n)?$ We get
\[
	\binom{n}{0}p^0q^n+\binom{n}{1}p^1q^{n-1}+\cdots +\binom{n}{n-1}p^{n-1}q^1+\binom{n}{n}p^nq^0=(p+q)^n=1
\]
so this does work.

Example: a lightbulb manufacturer sends bulbs in packs of 20 to stores. If production tolerances mean that $1/50$ of all bulbs are defective, what is the probability a pack up bulbs has more than 1 defective bulb in it?

What happens when $p=q?$ Then both must be $1/2$ and $p^kq^{n-k}=(1/2)^n.$ This puts us back in the coin flip model where all outcomes are equally likely. 

Some other notation:\\
If $X$ is a random variable with probability distribution which is a binomial distribution, we can write $X\sim \mathrm{Binom}(n,p)$ and say $X$ is distributed binomially with parameters $n$ and $p.$ For example, suppose you play your friend in a game but they win $3/5$ of the time. You decide to play 5 times. If $X$ is the number of times you win then we can say $X\sim \mathrm{Binom}(5,2/5).$
	
Concept questions:\\
\begin{itemize}
	\item What does a binomial distribution look like if $n=1$
	\item When $n$ becomes very large, what happends to $P(X=k).$
	\item If $X\sim \mathrm{Binom}(n,p)$ and $Y\sim \mathrm{Binom}{m,p}$ where $X$ and $Y$ are independent, describe the distribution for $X+Y.$
\end{itemize}

A tennis match consists of at most 5 sets, where the first player to 3 sets wins. If Nadal and Djokovic play then we estimate there is probability $11/20$ that Nadal will win a set. Assume the sets are independent.
\begin{enumerate}
	\item Find the probability Nadal wins in 5 sets.
	\item Find the probability Nadal wins in 4 sets.
	\item Find the probability Djokovic wins.
	\item Calculate the previous probability in another way (pretend to play all sets, Djokovic must win 3 or more)
\end{enumerate}

Question: What is the ``average'' outcome of one of these experiments. How do we define this?
\end{document}
