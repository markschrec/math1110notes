\documentclass[14,fleqn]{article}
\usepackage{amsmath}
\usepackage{amssymb}
\usepackage[top=.5 in,left=.5 in,right=.5 in,bottom=.5 in]{geometry}
\usepackage{enumerate}
\usepackage{ mathrsfs }
\usepackage{graphicx}
\usepackage{pgf,tikz}
\usepackage{mathrsfs}
\usepackage{gensymb}
\usepackage{venndiagram}
\usepackage{enumitem}
\usetikzlibrary{arrows}

\pagenumbering{gobble}

\setlength{\parindent}{0 pt}
\setlength{\parskip}{1 ex}

\newcommand{\lcm}{\textnormal{lcm}}
\newcommand{\norm}{\triangleabove right}
\newcommand{\bfm}[1]{$\boldsymbol{#1}$}
\newcommand{\Z}{\ensuremath{\mathbb{Z}}}
\newcommand{\R}{\ensuremath{\mathbb{R}}}
\newcommand{\C}{\ensuremath{\mathbb{C}}}
\renewcommand{\wedge}[1]{\ensuremath{\langle #1 \rangle}}
\newcommand{\infsum}[1]{\ensuremath{\sum_{n=#1}^\infty}}
\newcommand{\defn}[1]{\textbf{\underline{#1}}}
\newcommand{\var}{\ensuremath{\mathrm{Var}}}

%\begin{venndiagram3sets}[labelA=$S$,labelB=$T$,labelC=$U$]
%	\fillA
%	\fillOnlyC
%\end{venndiagram3sets}\\

%\begin{venndiagram2sets}[labelA=$S$,labelB=$T$]
%	\fillNotA
%	\fillNotB
%	\setpostvennhook{
%		\draw[] (labelAB) ++(0,-2.1) node {\raisebox{0pt}[0pt][0pt]{$(S\cap T)'$}};
%	}
%\end{venndiagram2sets}\\

\begin{document}
\section{Section 2.2/2.3: Matrices, Matrix Operations, and Systems of Linear Equations}

Before we start chapter 8 we need to do a little bit of review of matrices and matrix operations. Things to review:
\begin{enumerate}
	\item Definition of a matrix
	\item Matrix Addition
	\item Matrix Multiplication
	\item Identity Matrix
	\item Matrices as systems of equations
	\item Matrices acting on objects
\end{enumerate}

In real life experiments can be incredibly complicated. But there is one example which is fairly common and useful. Sometimes we have a sequence of experiments where the outcome of each depends only on the results of the previous experiment. Such a sequence of experiments is called a \defn{Markov Process}.

Examples:
\begin{itemize}
	\item A patient is being treated and their blood pressure is monitored. Each day their blood pressure is recored as low, normal, or high and they are treated accordingly.
	\item At UVA students can either be fine academically, be on academic probation, or suspension. Their academic status depends on their last two semesters.
	\item If we have a game with no draws, then each position is either a player 1 win or a player 2 win, and this only depends on the current position of the game.
\end{itemize}

When we have a Markov Process, we assume the experiment is performed at regular intervals, and the possible outcomes are always the same. These outcomes are called \defn{states}. The outcome of the current experiment is called the \defn{current state}. We can describe the outcomes of the experiments with tree diagrams or, even better, state diagrams.

\end{document}
