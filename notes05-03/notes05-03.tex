\documentclass[14,fleqn]{article}
\usepackage{amsmath}
\usepackage{amssymb}
\usepackage[top=.5 in,left=.5 in,right=.5 in,bottom=.5 in]{geometry}
\usepackage{enumerate}
\usepackage{ mathrsfs }
\usepackage{graphicx}
\usepackage{pgf,tikz}
\usepackage{mathrsfs}
\usepackage{gensymb}
\usepackage{venndiagram}
\usetikzlibrary{arrows}

\pagenumbering{gobble}

\setlength{\parindent}{0 pt}
\setlength{\parskip}{1 ex}

\newcommand{\lcm}{\textnormal{lcm}}
\newcommand{\norm}{\triangleleft}
\newcommand{\bfm}[1]{$\boldsymbol{#1}$}
\newcommand{\Z}{\ensuremath{\mathbb{Z}}}
\newcommand{\R}{\ensuremath{\mathbb{R}}}
\newcommand{\C}{\ensuremath{\mathbb{C}}}
\renewcommand{\wedge}[1]{\ensuremath{\langle #1 \rangle}}
\newcommand{\infsum}[1]{\ensuremath{\sum_{n=#1}^\infty}}
\newcommand{\defn}[1]{\textbf{\underline{#1}}}

%\begin{venndiagram3sets}[labelA=$S$,labelB=$T$,labelC=$U$]
%	\fillA
%	\fillOnlyC
%\end{venndiagram3sets}\\

%\begin{venndiagram2sets}[labelA=$S$,labelB=$T$]
%	\fillNotA
%	\fillNotB
%	\setpostvennhook{
%		\draw[] (labelAB) ++(0,-2.1) node {\raisebox{0pt}[0pt][0pt]{$(S\cap T)'$}};
%	}
%\end{venndiagram2sets}\\

\begin{document}
\section{Section 5.3: Venn Diagrams and counting}

Venn diagrams are very useful for finding the number of elements in various sets. We do this by looking at the number of elements in each \defn{Basic Region} of the Venn diagram.

\begin{venndiagram2sets}[labelNotAB=I,labelOnlyA=II,labelAB=III,labelOnlyB=IV]
\end{venndiagram2sets}
\begin{venndiagram3sets}[labelNotABC=VIII,labelOnlyA=II,labelOnlyB=III,labelOnlyC=IV,labelOnlyAB=V,labelOnlyBC=VI,labelOnlyAC=VII,labelABC=I]
\end{venndiagram3sets}

If we know the number of elements in ceach of these basic regions, we can find the number of elements in any region by simply adding.\\

Example: A survey of 126 first year students at UVA found that 32 will take a math class but not a history class, 28 will take a history class but not a math class, 58 will take a histor and a math class, while 8 will take neither.\\

How many students will take a math class?\\
How many will take a math or history class?\\
How many will not take a history class?\\

We will answer these questions by drawing the Venn diagram and labeling its basic regions. Let $M$ be the set of students taking a math class and $H$ be the set of students taking a history class.\\
\begin{venndiagram2sets}[labelNotAB=8,labelA=$M$,labelOnlyA=32,labelB=$H$,labelOnlyB=28,labelAB=58]
\end{venndiagram2sets}\\
To find how many students will take a math class we simply add up all of the regions that lie in $M$ this gives the number of students to take a math class is $n(M)=32+58=90.$

Similarly $n(M\cup H)=32+58+28=118.$ 

To find how many students will not take a history class, we simply add up all of the regions that don't lie in $H.$ So the number of students who will not take a history course is $n(H')=32+8=40.$

In the previous example, what do you notice about all of the numbers? When we do inclusion-exclusion, sometimes you have to subtract a number for the intersection. Why doesn't that happen here?\\

A useful fact: Using inclusion-exclusion, what is $n(A\cup A')?$ What is another way to write $A\cup A'?$ Putting these together we get the useful formula $n(U)=n(A)+n(A')$ for any set $A.$ This can make computations a little bit easier and will also come up when we do probability again.

Let's do one more example with 3 sets. An examination of the fortune 500 companies revealed this about their advertising practices:
\begin{itemize}
	\item 402 of the companies advertized with Television
	\item 364 of the companies advertized with Facebook
	\item 220 of the companies advertized with Radio
	\item 192 of the companies advertized with TV and Facebook
	\item 132 of the companies advertized with Facebook and Radio
	\item 203 of the companies advertized with TV and Radio
	\item 120 of the companies advertized with all 3
\end{itemize}

Answer the following questions:
\begin{enumerate}
	\item How many companies did not advertize with any of the 3? (21)
	\item How many companies advertized with just Radio? (5)
	\item How many companies advertized with Facebook and TV but not Radio? (172)
	\item How many companies advertized with exactly one of the 3? (92)
	\item How many companies did not advertize with all 3? (380)
\end{enumerate}

We start by drawing the Venn diagram with the basic regions. Work through the process with them working from the inside out. Don't forget to include the total complement.

This gives us the following Venn diagram\\
\begin{venndiagram3sets}[radius=2cm,overlap=1cm,labelA=TV,labelB=Facebook,labelC=Radio,labelOnlyA=27,labelOnlyB=60,labelOnlyC=5,labelOnlyAB=172,labelOnlyBC=12,labelOnlyAC=83,labelABC=120,labelNotABC=21]
\end{venndiagram3sets}

We can now read off the answers from the diagram (answers written with quesitons above in parentheses.)
\end{document}
